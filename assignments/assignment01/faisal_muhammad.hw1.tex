%% to produce a PDF copy, issue the following command:
%%
%%     pdflatex propositional-logic-examples.tex
%%
%% in the same directory containing the LaTeX style files:
%%
%%     prooftree.sty  and  boxproof.sty

\documentclass[11pt,leqno,fleqn]{article}

\usepackage{graphicx} 
\usepackage{times}              % better fonts for mathematical symbols
\usepackage{bm}                 % unlike \boldmath,
                                % \bm can be used anywhere within math mode
\usepackage[scaled=0.9]{helvet} % makes text a little smaller throughout,
                                % but not the text in math mode.
\usepackage{prooftree}
\usepackage{boxproof}

\setlength\hoffset{-5pt}      % horizontal offset, to move text horizontally
\setlength{\textwidth}{4.5in} % try different widths
\setlength\voffset{-5pt}      % vertical offset, to move text vertically
\setlength{\textheight}{7in}  % try different heights

\newcommand{\Hide}[1]{}             % use \Hide{bla bla} to hide ``bla bla''
\newcommand{\code}[1]{\texttt{#1}}  % use \code{...} to produce ASCII chars
\newcommand{\Intro}[1]{{#1}{\textrm{i}}}
\newcommand{\Elim}[1]{{#1}{\textrm{e}}}

\title{CS511 Formal Methods: Homework 01}
\author{Muhammad Faisal} 
\date{} % omit date

\begin{document}

\maketitle


\section{Problem One}

\subsection{DNF of majority function}
$\phi = (\lnot z \land y \land x) \lor (z \land \lnot y \land x) \lor (z \land y \land \lnot x) \lor (z \land y \land x) $

\subsection{CNF of majority function}
$\psi = (\lnot z \lor y \lor x) \land (z \lor \lnot y \lor x) \land (z \lor y \lor \lnot x) \land (z \lor y \lor x) $

\subsection{Checking with Z3 script}

\begin{verbatim}
   (declare-const x Bool)
   (declare-const y Bool)
   (declare-const z Bool)
   
   (declare-fun phi (Bool Bool Bool) Bool)
   (assert (= (phi x y z)
              (or (and x (and y (not z))) (or (and x (and (not y) z))
                (or (and (and (not x) y) z) (and (and x y) z) ) ))))
   
   (declare-fun psi (Bool Bool Bool) Bool)
   (assert (= (psi x y z)
              (and (or x (or y (not z))) (and (or x (or (not y) z))
                (and (or (or (not x) y) z) (or (or x y) z) ) ))))
   
   (assert ( = (phi x y z) (psi x y z)))
   
   (check-sat)
\end{verbatim}


\subsection{Checking with Z3Py script}
\begin{verbatim}
   from z3 import *

   x,y,z = Bools('x y z')

   phi = Or (And(Not(z),y,x), And(z,Not(y),x), And(z,y,Not(x)), And(z,y,x))

   psi = And (Or(Not(z),y,x), Or(z,Not(y),x), Or(z,y,Not(x)), Or(z,y,x)) 

   s = Solver()
   s.add((phi == psi))

   print s.check()
\end{verbatim}


\pagebreak

\section{Problem Two}
Prove the following theorems of propositional logic:

\subsection{$((P \to Q) \to Q) \to ((Q \to P) \to P)$}

% \begin{proofbox}
%     \[
%        \label{a1}\: ((P \to Q) \to Q)  \= \textrm{assume} \\
%        \: P\land Q    \= \Intro{\land}\ \ref{a1},\ref{a2} \\
%     \]
%     \: Q\to(P\land Q) \= \Intro{\to}  \\
%  \end{proofbox}



\subsection{$((P \to Q) \land (Q \to P)) \to ((P \lor Q) \to (P \land Q)) $}

\begin{proofbox}
    \[
       \label{a1}\: (P \to Q) \land (Q \to P)  \= \textrm{assume} \\
       \label{a2}\: P \to Q    \= \Elim{\land}_1\ \ref{a1}\\
       \label{a3}\: Q \to P    \= \Elim{\land}_2\ \ref{a1}\\
       \[
            \label{a4}\: P \lor Q  \= \textrm{assume} \\
            \[
                \label{a5}\: P    \= \Elim{\lor}_1 \ref{a4}\\
                \label{a6}\: Q    \= \Elim{\to}\ \ref{a2}, \ref{a5}\\
                \label{a7}\: P \land Q    \= \Intro{\land}\ \ref{a5},\ref{a6} \\
            \]
            \[
                \label{a8}\: Q    \= \Elim{\lor}_2 \ref{a4}\\
                \label{a9}\: P    \= \Elim{\to}\ \ref{a3}, \ref{a8}\\
                \label{a10}\: P \land Q    \= \Intro{\land}\ \ref{a8},\ref{a9} \\
            \]
            \: P\land Q    \= \Intro{\land}\ \ref{a4},\ref{a5}-\ref{a7},\ref{a8}-\ref{a10} \\
       \]
       \: ((P \lor Q) \to (P \land Q))    \= \Elim{\lor}\ \ref{a1},\ref{a2} \\
    \]
    \: ((P \to Q) \land (Q \to P)) \to ((P \lor Q) \to (P \land Q))  \= \Intro{\to}  \\
 \end{proofbox}




\subsection{$(P \to Q) \to ( (\lnot P \to Q) \to Q)$}
\begin{proofbox}
    \[
       \label{a1}\: P \to Q  \= \textrm{assume} \\
       \[
            \label{a2}\: \lnot P \to Q  \= \textrm{assume} \\
            \[
                \label{a3}\: \lnot Q    \= \textrm{assume}\\
                \label{a4}\: \lnot \lnot P    \= \textrm{MT} \ \ref{a2},\ref{a3}\\
                \label{a5}\: P    \= \Elim{\neg{\neg{}}} \ \ref{a4}\\
                \label{a6}\: Q    \= \Elim{\to}\ \ref{a1}, \ref{a5}\\
                \label{a7}\: \lnot \lnot Q    \= \Intro{}{\neg{\neg}} \ \ref{a6}\\
                \label{a8}\: \bot    \= \Elim{\neg}\ \ref{a3}, \ref{a7}\\
            \]
            \: Q    \= \textrm{PBC} \ \ref{a3}-\ref{a8} \\
       \]
       \: (\lnot P \to Q) \to Q    \= \Intro{\to}  \ \ref{a2}-\ref{a9}\\
    \]
    \: (P \to Q) \to ( (\lnot P \to Q) \to Q)  \= \Intro{\to}  \\
 \end{proofbox}



\pagebreak



\section{Problem 04}

\begin{enumerate}
   \item $\Intro{\lnot}$ : $\phi \to \bot$
   \item $\Elim{\lnot}$ : $\frac {\phi} {\bot \to \phi}$
   \item $\Intro{\lnot{\lnot}}$ : $(\phi \to \bot) \to  \bot$
   \item $\Elim{\lnot{\lnot}}$ : $\frac{\phi} {(\bot \to \phi) \to \phi} $
\end{enumerate}


\section{Problem 05}
\subsection{adequate sets}
   The following explain why which one can be adequeate set.
\begin{enumerate}
   \item $\{\lnot \ , \land\}$: $\phi \to \psi \equiv \lnot (\phi \land \lnot \psi)$, $ ( \phi \land \psi ) \equiv (\lnot (\lnot \phi \lor \lnot \psi)) $
   \item $\{\lnot \ , \to\}$: $(\phi \land \psi) \equiv (\phi \to \psi, \psi \to phi)$,  $(\phi \lor \psi) \equiv \phi \to \lnot \psi$
   \item $\{\to \ , \bot\}$: $\lnot p \equiv p \to \bot $, $(\phi \lor \lnot \psi) \equiv \phi \to \psi$ 
\end{enumerate}

\subsection{b}
Without either $\lnot$, $\bot$, the symbols can not express all situations. For example, the truth table for such expressions with these symbols would have only the case when all variables are set to 1 because negation is not a possible case for any variable.

\subsection{C}
They are not adequate because they can only express tautologies. For example, the $p \to q$ can not be expressed using these symbols.


% \section{Problem 06}

% \section{Example: $P \vdash Q\to(P\land Q)$}

% \begin{proofbox}
%    \label{a1}\: P \= \textrm{premise} \\
%    \[
%       \label{a2}\: Q  \= \textrm{assume} \\
%       \: P\land Q    \= \Intro{\land}\ \ref{a1},\ref{a2} \\
%    \]
%    \: Q\to(P\land Q) \= \Intro{\to}  \\
% \end{proofbox}

% \section{Example: $P\to(Q\to R) \vdash P\land Q\to R$}

% \begin{proofbox}
%    \label{b1}\: P\to(Q\to R) \= \textrm{premise} \\
%    \[
%       \label{b2}\: P\land Q    \= \textrm{assume} \\
%       \label{b3}\: P \= {\Elim{\land}}_1\ \ \ref{b2}\\
%       \label{b4}\: Q\to R \= \Elim{\to}\ \ \ref{b1},\ref{b3}\\
%       \label{b5}\: Q \= {\Elim{\land}}_2\ \ \ref{b2}\\
%       \: R \=\Elim{\to}\ \ \ref{b4},\ref{b5}\\
%    \]
%       \: P\land Q\to R \= \Intro{\to} \\
% \end{proofbox}

% \section{Example: $P\land Q\to R \vdash P\to(Q\to R)$}

% \begin{proofbox}
%    \label{c1}\: P\land Q\to R \= \textrm{premise} \\
%    \[
%       \label{c2}\: P     \= \textrm{assume} \\
%       \[
%          \label{c3}\: Q  \= \textrm{assume} \\
%          \label{c4}\: P\land Q \= \Intro\land\ \ref{c2},\ref{c3}\\
%          \: R \= \Elim\to\ \ref{c1},\ref{c4}
%       \]
%       \: Q\to R \= \Intro\to \\
%    \]
%    \: P\to(Q\to R) \= \Intro\to \\
% \end{proofbox}

% \section{Example: $P\to(Q\to R) \vdash (P\to Q)\to(P\to R)$}

% \begin{proofbox}
%    \label{d1}\: P\to(Q\to R) \= \textrm{premise} \\
%    \[
%       \label{d2}\: P\to Q  \= \textrm{assume} \\
%       \[
%          \label{d3}\: P    \= \textrm{assume} \\
%          \label{d4}\: Q \= \Elim\to\ \ref{d2},\ref{d3}\\
%          \label{d5}\: Q\to R \= \Elim\to \ref{d1},\ref{d3}\\
%          \: R \= \Elim\to\ \ref{d5},\ref{d4}\\
%       \]
%       \: P\to R \= \Intro\to \\
%    \]
%    \: (P\to Q)\to(P\to R) \= \Intro\to \\
% \end{proofbox}

% \section{Example:  $\quad P\land\neg{Q}\to R,\ \ \neg{R},\ \ P \quad\vdash\quad Q$}

% \begin{proofbox}
%    \label{aa}\: P\land\neg{Q}\to R \= \textrm{premise}\\
%    \label{bb}\: \neg{R} \= \textrm{premise}\\
%    \label{cc}\: P \= \textrm{premise}\\
%    \[
%       \label{dd}\: \neg{Q} \= \textrm{assume}\\
%       \label{ee}\: P\land\neg{Q} \= {\Intro{\land}}\ \ \ref{cc},\ref{dd}\\
%       \label{ff}\: R \= \Elim{\to}\ \ \ref{aa},\ref{ee}\\
%       \label{gg}\: \bot \= {\Elim{\neg{}}}\ \ \ref{ff},\ref{bb}\\
%    \]
%       \label{hh}\: \neg{\neg{Q}} \= \Intro{\neg{}} \\
%       \: Q \= \Elim{\neg{\neg{}}}\ \ \ref{hh} \\
% \end{proofbox}




\end{document}

